\documentclass[fleqn, 14pt]{extarticlej}
\oddsidemargin=-1cm
\textwidth=18cm
\textheight=23cm
\topmargin=0cm
\headheight=1cm
\headsep=0cm
\footskip=0.8cm

\pagestyle{plain}
\usepackage{ulem}
\usepackage{indentfirst}

\renewcommand\labelenumi{(\theenumi)}
\def\descriptionlabel#1{\hspace\labelsep #1}

\makeatletter
\def\section{\@startsection {section}{1}{\z@}%
{-3.5ex \@plus -1ex \@minus -.2ex}%
{2.3ex \@plus.2ex}%
{\sectionformat}}
\def\sectionformat{\normalsize}
\makeatother

\def\l@section{\@dottedtocline{1}{2.5em}{2.3em}}
% 最初の 1 は toc の深さの定義、1.5em は左端からの字下げ幅、
% 2.3em は番号と見出しの間隔
\def\l@subsection{\@dottedtocline{2}{3.8em}{3.2em}}

\begin{document}
\setcounter{section}{-1}

%%%%%%%%%%%%%%%%%%%%%%%%%%%%
%% 表題
%%%%%%%%%%%%%%%%%%%%%%%%%%%%
\begin{center}
  {\Large {\bf 記録書} No.2}

  (2013年 4月 15日 ~ 5月 7日)
\end{center}
\begin{flushright}
  2013年 5月 8日\\
  乃村研究室   河野 達生
\end{flushright}

\vspace{-1.5em}
%%%%%%%%%%%%%%%%%%%%%%%%%%%%
%% 前回ミーティングでの指導・指摘事項
%%%%%%%%%%%%%%%%%%%%%%%%%%%%
\section*{0. 前回ミーティングでの指導・指摘事項}
\vspace{-0.5em}
\begin{tabular}{p{18zw}lr}
  (1) 記録書を書いてある通りに読む.&[4/15,全体ミーティング,谷口先生]
\end{tabular}
\\
\vspace{-1.5em}
%%%%%%%%%%%%%%%%%%%%%%%%%%%%
%% 実績
%%%%%%%%%%%%%%%%%%%%%%%%%%%%
\section*{1. 実績}
\vspace{-1.0em}

\begin{description}
\itemsep -1mm

  \item[ 1.1] 研究関連
  
    \begin{tabular}{p{11.8cm}l}
      (1) 平成25年度GNグループ新人研修課題 \\
		\hspace{0.5zw}(A) RubyによるTwitterBotプログラムの作成 &(100%, +0%) \\
		\hspace{0.5zw}(B) Ruby on Railsによる商品管理プログラムの作成&(100%, +100%) \\
        \hspace{0.5zw}(C) TwitterBotプログラムの仕様書の作成 &(100%, +100%) \\
        \hspace{0.5zw}(D) 課題の報告書の作成&(100% ,+100%) \\
      (2) 論文[1]要約 &(100%, +100%)\\
	  (3) 開発のワークフローの読解 &(50%, +50%)

    \end{tabular}

  \item[ 1.2] 研究室関連
  
    \begin{tabular}{p{11.8cm}l}
      (1) Ruby on Rails 勉強会&(4/15)\\
	  (2) 全体ミーティング&(4/15)\\
      (3) 第125回GN検討打合せ & (4/19)\\
      (4) 第21回乃村杯 & (5/2)\\

    \end{tabular}

  \item[ 1.3] 大学関連
  
    \begin{tabular}{p{11.8cm}l}
      (1) 平成25年度津島地区学生定期健康診断&(4/22)\\
     
    \end{tabular}
    
    \item[ 1.4] 就職活動
    
	   \begin{tabular}{p{11.8cm}l}
		 (1) 株式会社NTTデータ四国(三次選考)&(4/17)\\
		 (2) 株式会社富士通システムズ・ウエスト(最終選考)&(4/19)\\
		 (3) 日本ユニシス株式会社(一次選考)&(4/19)\\
		 (4) 株式会社NTTデータ(最終選考)&(4/22)\\
		 
       \end{tabular}

\end{description}

\vspace{-1.5em}
%%%%%%%%%%%%%%%%%%%%%%%%%%%%
%% 詳細と感想
%%%%%%%%%%%%%%%%%%%%%%%%%%%%
\section*{2. 詳細と感想}
\vspace{-1.0em}

\begin{description}
  \itemsep -1mm

\item[ 2.1] 研究関連
  \vspace{-0.8em}
  \begin{enumerate}
    \renewcommand{\labelenumi}{(1-B)}
  \item GNグループ新人研修課題の商品管理プログラムの作成において,参考文献[2]の内容に従い課題を進めていった.このため,どこのファイルを変更するとどこに反映されるかの理解はできたが,プログラムの内容については理解が浅いと感じる.Ruby on RailsはLastNoteの開発で使用するため,復習する.
  \end{enumerate}

  \vspace{-0.5em}
  \item[ 2.2] 研究室関連
    \vspace{-1.0em}
    \begin{enumerate}
      \renewcommand{\labelenumi}{(3)}
    \item 乃村研究室所属後初の乃村杯に参加した.種目は囲碁であり,岡山後楽園の観騎亭で行われた.観騎亭の趣のある中での対局であった.初めて囲碁を行い,初めて岡山後楽園へ行き,初めて乃村杯へ参加し,初めて尽くしの一日だった.
    
%     一回戦では北川さんに勝つことができた.
    \end{enumerate}
    
  \vspace{-0.5em}
  \item[ 2.4] 就職活動関連
    \vspace{-1.0em}
    \begin{enumerate}
      \renewcommand{\labelenumi}{(4)}
    \item 第一志望の株式会社NTTデータの選考が不合格であり,就職活動のモチベーションが大きく下がってしまった.就職活動を続けるか,大学院に進学するか考える.4月下旬に最終選考まで進んでいた企業から不合格通知を頂き,将来が不安になった.内々定を頂いた企業もあるが親や自分のやりたいことを考えた結果,辞退した.しかし,今になりもう少し考えればよかったと後悔している.

    
    \end{enumerate}

\end{description}

\vspace{-1.5em}
%%%%%%%%%%%%%%%%%%%%%%%%%%%%
%% 予定
%%%%%%%%%%%%%%%%%%%%%%%%%%%%
\section*{3. 予定}
\vspace{-0.8em}

\begin{description}
  \itemsep -1mm
\item[ 3.1] 研究関連

  \begin{tabular}{p{11.8cm}l}
    (1) 開発ワークフローの読解&(5/10)
    
  \end{tabular}

\item[ 3.2] 研究室関連

  \begin{tabular}{p{11.8cm}l}
    (1) 第70回GN開発打合せ&(5/10)\\
    (2) 第3回TOEIC勉強会&(5/14)\\
	(3) 全体ミーティング&(5/20)\\ 
    
  \end{tabular}

\item[ 3.3] 大学関連

  \begin{tabular}{p{11.8cm}l}
    (1) 第180回TOEIC公開テスト&(5/26)\\
    (2) カレッジTOEIC&(6/1)\\
  \end{tabular}
  
  \end{description}
  
\section*{4. その他}
5/26と6/1にTOEICテストがある.英語力を上げるためウォークマンにリスニングのCDを入れ,常に聞いて慣れておく.

\begin{thebibliography}{99}
  \bibitem {book1} 三原俊介,谷口秀夫,乃村能成,南裕也:作業発生の規則性を扱うカレンダシステムの評価,研究報告マルチメディア通信と分散処理(DPS),Vol.2012-DPS-150,No.46,pp.1-6(2012).
  \bibitem {book2} Ruby, S., Thomas, D., Hansson,D. et al(著),前田修吾(訳):RailsによるアジャイルWebアプリケーション開発 第4版, pp.55-109, オーム社(2011).
\end{thebibliography}

\end{document}

%\vspace{-1.5