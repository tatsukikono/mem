\documentclass[fleqn, 14pt]{extarticlej}
\oddsidemargin=-1cm
\textwidth=18cm
\textheight=23cm
\topmargin=0cm
\headheight=1cm
\headsep=0cm
\footskip=0.8cm

\pagestyle{plain}
\usepackage{ulem}
\usepackage{indentfirst}

\renewcommand\labelenumi{(\theenumi)}
\def\descriptionlabel#1{\hspace\labelsep #1}

\makeatletter
\def\section{\@startsection {section}{1}{\z@}%
{-3.5ex \@plus -1ex \@minus -.2ex}%
{2.3ex \@plus.2ex}%
{\sectionformat}}
\def\sectionformat{\normalsize}
\makeatother

\def\l@section{\@dottedtocline{1}{2.5em}{2.3em}}
% 最初の 1 は toc の深さの定義、1.5em は左端からの字下げ幅、
% 2.3em は番号と見出しの間隔
\def\l@subsection{\@dottedtocline{2}{3.8em}{3.2em}}

\begin{document}
\setcounter{section}{-1}

%%%%%%%%%%%%%%%%%%%%%%%%%%%%
%% 表題
%%%%%%%%%%%%%%%%%%%%%%%%%%%%
\begin{center}
  {\Large {\bf 記録書} No.6}

  (2013年 6月 17日〜 6月 28日)
\end{center}
\begin{flushright}
  2013年 7月 1日\\
  乃村研究室\\
  河野 達生
\end{flushright}


%%%%%%%%%%%%%%%%%%%%%%%%%%%%
%% 前回ミーティングでの指導・指摘事項
%%%%%%%%%%%%%%%%%%%%%%%%%%%%
\section*{0. 前回ミーティングからの指導・指摘事項}
\vspace{-0.5em}
\begin{tabular}{p{21zw}l}
  特になし
\end{tabular}


%%%%%%%%%%%%%%%%%%%%%%%%%%%%
%% 実績
%%%%%%%%%%%%%%%%%%%%%%%%%%%%
\section*{1. 実績}


\begin{description}
\vspace{-0.5em}
  \item[ 1.1] 研究関連

    \begin{tabular}{p{11.8cm}l}
      (1) タスクの整理の定義&(30%,+30%)\\
    \end{tabular}
\vspace{-0.5em}
  \item[ 1.2] 研究室関連

    \begin{tabular}{p{11.8cm}l}
      (1) 第72回GN開発打合せ&(6/19)\\
      (2) 平成25年度M2,D論文紹介&(6/24)\\
      (3) 平成25年度乃村研勉強会&(6/27)\\
      (4) 第129回GN検討打合せ&(6/27)\\
      (5) DICOMO2013発表練習&(6/28)\\
    \end{tabular}
\vspace{-0.5em}
  \item[ 1.3] 大学関連

    \begin{tabular}{p{11.8cm}l}
      (1) 情報化における職業&(6/21)\\
    \end{tabular}
\end{description}


%%%%%%%%%%%%%%%%%%%%%%%%%%%%
%% 詳細と感想
%%%%%%%%%%%%%%%%%%%%%%%%%%%%
\section*{2. 詳細と感想}
\vspace{-0.5em}
\begin{description}

\vspace{-0.5em}
\item[ 2.1] 研究関連

  \begin{enumerate}
    \renewcommand{\labelenumi}{(1)}
  \item カレンダに整理の機能を追加するため,整理の定義を考察している.予定の名前や開始時刻のカレンダの情報のどのように整理するかを考察している.同じ条件で整理を行う際に,人によって差が生じないような基準を定める.また,情報の可視化においては,GN検討打合せの度にアイデアが出てくるので,実装できるかどうかは別として想像を膨らませていくのは楽しい.
  \end{enumerate}
  
  \newpage
\vspace{-0.5em}
 \item[ 2.2] 研究室関連

    \begin{enumerate}
      \renewcommand{\labelenumi}{(1)}
    \item 第72回GN開発打合せに参加した.LastNoteのテスト期間が終了しリリースを行った.結果は失敗だった.失敗した原因について調査していたが,私は知識がないためその場にいるだけとなっていた.話を聞きながらわからない単語をメモすることで理解を深めていく.
      \renewcommand{\labelenumi}{(2)}
    \item 平成25年度M2,D論文紹介を聴講した.発表内容は理解できない点もあった.しかし,発表の仕方について参考になる点が多くあった.具体的には,章立ての方法や図の使い方である.今回学んだ点を参考に,自信が発表するときに取り入れていきたい.
    \end{enumerate}

\end{description}


%%%%%%%%%%%%%%%%%%%%%%%%%%%%
%% 予定
%%%%%%%%%%%%%%%%%%%%%%%%%%%%
\section*{3. 予定}

\vspace{-0.5em}
\begin{description}
\vspace{-0.5em}
\item[ 3.1] 研究関連

  \begin{tabular}{p{11.8cm}l}
	(1) タスクの整理の定義&(7/11)\\
    
  \end{tabular}
\vspace{-0.5em}
\item[ 3.2] 研究室関連

  \begin{tabular}{p{11.8cm}l}
    (1) 暑気払い&(7/4)\\
    (2) 平成25年度乃村研勉強会&(7/8)\\
    (3) 第130回GN検討打合せ&(7/11)\\
    (4) 全体ミーティング&(7/22)\\
  \end{tabular}
\vspace{-0.5em}
\item[ 3.3] 大学関連

  \begin{tabular}{p{11.8cm}l}
   (1) カレッジTOEIC&(7/6)\\

  \end{tabular}

\end{description}

%\vspace{-1.5em}
%%%%%%%%%%%%%%%%%%%%%%%%%%%%
%% その他
%%%%%%%%%%%%%%%%%%%%%%%%%%%%
\vspace{-0.5em}
\section*{4. その他}
私はかに道楽でフロントスタッフのアルバイトをしている.夏が近づきかにの旬でない時期になり,フロントでお客様を待っている時間が増えた.この時間には,将来のことを考えることが多い.かに道楽のお客様には会社の会長や社長の常連が多いので,どんな生き方・考え方をしてきたか聞いてみたい.そして自分が人生の分岐点に立ったときの参考にしたい.また,現在かに道楽では期間限定でかにすきの夏バージョン「夏すき」を販売している.

\end{document}
