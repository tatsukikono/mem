\documentclass[fleqn, 14pt]{extarticlej}
\oddsidemargin=-1cm
\textwidth=18cm
\textheight=23cm
\topmargin=0cm
\headheight=1cm
\headsep=0cm
\footskip=0.8cm

\pagestyle{plain}
\usepackage{ulem}
\usepackage{indentfirst}

\renewcommand\labelenumi{(\theenumi)}
\def\descriptionlabel#1{\hspace\labelsep #1}

\makeatletter
\def\section{\@startsection {section}{1}{\z@}%
{-3.5ex \@plus -1ex \@minus -.2ex}%
{2.3ex \@plus.2ex}%
{\sectionformat}}
\def\sectionformat{\normalsize}
\makeatother

\def\l@section{\@dottedtocline{1}{2.5em}{2.3em}}
% 最初の 1 は toc の深さの定義、1.5em は左端からの字下げ幅、
% 2.3em は番号と見出しの間隔
\def\l@subsection{\@dottedtocline{2}{3.8em}{3.2em}}

\begin{document}
\setcounter{section}{-1}

%%%%%%%%%%%%%%%%%%%%%%%%%%%%
%% 表題
%%%%%%%%%%%%%%%%%%%%%%%%%%%%
\begin{center}
  {\Large {\bf 記録書} No.5}

  (2013年 6月 3日〜 6月 14日)
\end{center}
\begin{flushright}
  2013年 6月 17日\\
  乃村研究室\\
  河野 達生
\end{flushright}

\vspace{-1.5em}
%%%%%%%%%%%%%%%%%%%%%%%%%%%%
%% 前回ミーティングでの指導・指摘事項
%%%%%%%%%%%%%%%%%%%%%%%%%%%%
\section*{0. 前回ミーティングからの指導・指摘事項}
\vspace{-0.5em}
\begin{enumerate}
  \item 打合せに提出する資料は,あらかじめ先輩に見ていただき,研究の方向性を議論する.             [6/13,第128回検討打合せ,乃村先生]
\end{enumerate}
\\

\vspace{-1.5em}
%%%%%%%%%%%%%%%%%%%%%%%%%%%%
%% 実績
%%%%%%%%%%%%%%%%%%%%%%%%%%%%
\section*{1. 実績}
\vspace{-1.0em}

\begin{description}
  \itemsep -1mm

  \item[ 1.1] 研究関連

    \begin{tabular}{p{11.8cm}l}
      (1) タスクの表示方法の変更&(20%,+20%)\\
	  (2) タスクの整理の定義&(0%,0%)\\
	  
    \end{tabular}

  \item[ 1.2] 研究室関連

    \begin{tabular}{p{11.8cm}l}
      (1) 第1回平成25年度乃村研勉強会&(6/3)\\
      (2) 第71回GN開発打合せ&(6/4)\\
      (3) 第128回GN検討打合せ&(6/13)\\    
    
    \end{tabular}

  \item[ 1.3] 大学関連

    \begin{tabular}{p{11.8cm}l}
      (1) 情報化における社会&(6/7,6/14)\\

    \end{tabular}

  \item[ 1.4] 就職活動

    \begin{tabular}{p{11.8cm}l}
      (1) 野村総合研究所&(6/10)\\

    \end{tabular}

\end{description}

\vspace{-1.5em}
%%%%%%%%%%%%%%%%%%%%%%%%%%%%
%% 詳細と感想
%%%%%%%%%%%%%%%%%%%%%%%%%%%%
\section*{2. 詳細と感想}
\vspace{-1.0em}

\begin{description}
  \itemsep -1mm

\item[ 2.1] 研究関連
  \vspace{-0.8em}
  \begin{enumerate}
    \renewcommand{\labelenumi}{(1)}
  \item EnCalにタスクの一覧を表示する機能を追加している.現在,タスクの一覧を表示させるページを作成した.作成したページにおいてタスクを更新時刻順に表示ができる.今後,EnCalで簡単にミッションやリカーレンスを作成できる表示方法を検討する.\\
  
	\renewcommand{\labelenumi}{(2)}
  \item 
タスクの一覧を表示させる上で,タスクが整理された状態と整理されてない状態の定義を明確にする.このため,実際にEnCalを使用し,ミッションやリカーレンスを作成する際の問題や見つける際の問題を考察する.\\
  
  \end{enumerate}
\newpage
  \vspace{-0.5em}
  \item[ 2.2] 研究室関連
    \vspace{-1.0em}
    \begin{enumerate}
      \renewcommand{\labelenumi}{(1)}
    \item 第1回平成25年度乃村研勉強会に参加した.乃村研勉強会では,Bad Data Handbook[1]を読解する.毎週持ち回りで担当箇所を要約する.第1回勉強会では既存のデータからコマンドラインやスクリプトを用いて,自分の欲しい情報を見つけ出す方法を解説していた.私は,catやgrepといったコマンドを使うことに慣れていないため,今後コマンドの種類を把握する.自身の研究を進めていく際にも利用し,作業効率を上げる.
    
    
    \end{enumerate}

	%\vspace{-0.5em}
%% \item[ 2.3] 大学関連
%% 	\vspace{-0.8em}
%% 	\begin{enumerate}
%% 	  \renewcommand{\labelenumi}{(1)}
%% 	\item 計画停電の復旧作業に参加した.
%% \end{enumerate}
\end{description}

\vspace{-1.5em}
%%%%%%%%%%%%%%%%%%%%%%%%%%%%
%% 予定
%%%%%%%%%%%%%%%%%%%%%%%%%%%%
\section*{3. 予定}
\vspace{-0.8em}

\begin{description}
  \itemsep -1mm
\item[ 3.1] 研究関連

  \begin{tabular}{p{11.8cm}l}
	(1) タスクの一覧表示&(7月上旬)\\
    (2) EnCalのUIの考察&(6/27)\\
	      
  \end{tabular}

\item[ 3.2] 研究室関連

  \begin{tabular}{p{11.8cm}l}
	(1) 第72回GN開発打合せ&(6/19)\\
	(2) 平成25年度乃村研勉強会&(6/27,7/1)\\
	(3) 第129回GN検討打合せ&(6/27)\\
	(4) 乃村研ミーティング&(7/1)\\

  \end{tabular}

\item[ 3.3] 大学関連

  \begin{tabular}{p{11.8cm}l}
	(1) カレッジTOEIC申込み締切&(6/19)\\
  \end{tabular}

\end{description}

%\vspace{-1.5em}
%%%%%%%%%%%%%%%%%%%%%%%%%%%%
%% その他
%%%%%%%%%%%%%%%%%%%%%%%%%%%%
\section*{4. その他}
\vspace{-1.0em}
以前からMacbookProの購入を検討していた.6月10日のWWDC2013で新しいMacbookProが発表される噂があったため,WWDC2013後に購入しようと考えていた.しかし,今回は新しいMacbookProの発表がなかったため,昨年の秋に発売されたMacbookProを購入した.乃村研の先輩方が,Macを有効に活用しているので,私も宝の持腐れにならないように活用する.

\section*{5. 関連文献}
\vspace{-1.0em}
	[1] McCallum, E.Q.: Bad Data Handbook,O'REILLY(2012).

% \vspace{-1.0em}
% \section*{5. 研究関連文献}
% \vspace{-1.0em}

\end{document}
