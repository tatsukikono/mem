\documentclass[fleqn, 14pt]{extarticlej}
\oddsidemargin=-1cm
\textwidth=18cm
\textheight=23cm
\topmargin=0cm
\headheight=1cm
\headsep=0cm
\footskip=0.8cm

\pagestyle{plain}
\usepackage{ulem}
\usepackage{indentfirst}

\renewcommand\labelenumi{(\theenumi)}
\def\descriptionlabel#1{\hspace\labelsep #1}

\makeatletter
\def\section{\@startsection {section}{1}{\z@}%
{-3.5ex \@plus -1ex \@minus -.2ex}%
{2.3ex \@plus.2ex}%
{\sectionformat}}
\def\sectionformat{\normalsize}
\makeatother

\def\l@section{\@dottedtocline{1}{2.5em}{2.3em}}
% 最初の 1 は toc の深さの定義、1.5em は左端からの字下げ幅、
% 2.3em は番号と見出しの間隔
\def\l@subsection{\@dottedtocline{2}{3.8em}{3.2em}}

\begin{document}
\setcounter{section}{-1}

%%%%%%%%%%%%%%%%%%%%%%%%%%%%
%% 表題
%%%%%%%%%%%%%%%%%%%%%%%%%%%%
\begin{center}
  {\Large {\bf 記録書} No.4}

  (2013年 5月 18日〜 6月 2日)
\end{center}
\begin{flushright}
  2013年 6月 3日\\
  乃村研究室\\
  河野 達生
\end{flushright}

\vspace{-1.5em}
%%%%%%%%%%%%%%%%%%%%%%%%%%%%
%% 前回ミーティングでの指導・指摘事項
%%%%%%%%%%%%%%%%%%%%%%%%%%%%
\section*{0. 前回ミーティングでの指導・指摘事項}
\vspace{-0.5em}
\begin{tabular}{p{21zw}l}
  (1)特になし
\end{tabular}
\\

\vspace{-1.5em}
%%%%%%%%%%%%%%%%%%%%%%%%%%%%
%% 実績
%%%%%%%%%%%%%%%%%%%%%%%%%%%%
\section*{1. 実績}
\vspace{-1.0em}

\begin{description}
  \itemsep -1mm

  \item[ 1.1] 研究関連

    \begin{tabular}{p{11.8cm}l}
      (1) EnCalの概要把握&(100%,+90%)\\
	  (2) 名前順にタスクを表示する機能の作成&(0%,+0%)\\
    \end{tabular}

  \item[ 1.2] 研究室関連

    \begin{tabular}{p{11.8cm}l}
      (1) 全体ミーティング&(5/20)\\
      (2) 研究室部屋別対抗ボウリング大会&(5/20)\\
      (3) 第127回GN検討打合せ&(5/30)\\

    \end{tabular}

  \item[ 1.3] 大学関連

    \begin{tabular}{p{11.8cm}l}
    (1) 情報化における職業&(5/24)\\
	(2) 第180回TOEIC公開テスト&(5/26)\\
    (3) カレッジTOEIC&(6/1)\\
    
	\end{tabular}
  
  \item[ 1.4] 就職関連

    \begin{tabular}{p{11.8cm}l}
	  (1)三菱UFJインフォメーションテクノロジー株式会社\\
	 (一次選考)&(5/11)\\
	  (2) キヤノンITソリューションズ(一次選考)&(5/27)\\
    \end{tabular}
    

\end{description}

\vspace{-1.5em}
%%%%%%%%%%%%%%%%%%%%%%%%%%%%
%% 詳細と感想
%%%%%%%%%%%%%%%%%%%%%%%%%%%%
\section*{2. 詳細と感想}
\vspace{-1.0em}

\begin{description}
  \itemsep -1mm

\item[ 2.1] 研究関連
  \vspace{-0.8em}
  \begin{enumerate}
    \renewcommand{\labelenumi}{(1)}
  \item EnCalを使用し,改善案をまとめた.EnCalとは,作業の周期性や関連性を扱うカレンダシステムである.自身の提案した改善案はチケットとしてRedmineに登録した.私はコードを書くのが苦手である.私は,習うより慣れろのように体で覚えていくタイプなので,簡単なコードから書きはじめ,苦手意識をなくし徐々に理解を深めていきたい.そして,自身が提案した改善案を実装する.\\
    
  \end{enumerate}

  \vspace{-0.5em}
  \item[ 2.2] 研究室関連
    \vspace{-1.0em}
    \begin{enumerate}
      \renewcommand{\labelenumi}{(2)}
    \item 部屋別ボウリング大会に参加した.ルールは,2ゲームの合計点数の平均を部屋別で競うものである.106号室のメンバは,ボウリングが苦手な人が多いと聞いていたが,謙遜していると思っていた.結果は最下位であり,3位とアベレージが30ピン差あったので謙遜ではなかったのだと思った.結果は,最下位だったが個人的には楽しむことができた.また今回は,乃村研の参加者が少なかったため,次回は全員で参加して結果を残したい.
    
    \end{enumerate}

\vspace{-0.5em}
\item[ 2.3] 大学関連
\vspace{-0.8em}
	\begin{enumerate}
	  \renewcommand{\labelenumi}{(2)}
	\item 第180回TOEIC公開テストを受験した.久しぶりのTOEICであったため,少し緊張した.しかし,リスニングは聞き取れた感触があった.SWLABのB4で行われているTOEIC勉強会の成果が出たのではないかと思う.
	\end{enumerate}

\end{description}

\vspace{-1.5em}
%%%%%%%%%%%%%%%%%%%%%%%%%%%%
%% 予定
%%%%%%%%%%%%%%%%%%%%%%%%%%%%
\section*{3. 予定}
\vspace{-0.8em}

\begin{description}
  \itemsep -1mm
  
\item[ 3.1] 研究関連
  
  \begin{tabular}{p{11.8cm}l}
    (1) タスクの一覧の表示&(6月下旬)\\
     
  \end{tabular}
  
\item[ 3.2] 研究室関連

  \begin{tabular}{p{11.8cm}l}
	(1) 第71回開発打合せ&(6/4)\\
    (2) 平成25年度乃村研勉強会&(6/10)\\
    (3) 第128回GN検討打合せ&(6/13)\\    
    
  \end{tabular}

\item[ 3.3] 大学関連

  \begin{tabular}{p{11.8cm}l}
    (1) 情報化における社会&(6/7)\\
  \end{tabular}

\end{description}

%\vspace{-1.5em}
%%%%%%%%%%%%%%%%%%%%%%%%%%%%
%% その他
%%%%%%%%%%%%%%%%%%%%%%%%%%%%
\section*{4. その他}
\vspace{-1.0em}
最近,自身のノートパソコンを初期化した.先輩方がノートパソコンを自身の研究に活用していることにあこがれ,私もノートパソコンに研究室で使用している文書作成用計算機と同じような環境構築をし,家でも作業できるようにしたい.また,新しいノートパソコンの購入も検討中である.気になっているのは,東芝のdynabook KIRA,NECのLaVie ZもしくはAppleのMacBookである.Windowsに慣れているため,Macを購入するのにためらってしまう.乃村研は,Mac使用者が多いため,アドバイスを頂きたい.



% \section*{5. 研究関連文献}
% \vspace{-1.0em}
% [1]三原俊介,谷口秀夫,乃村能成,南裕也:作業発生の規則性を扱うカレンダシステムの評価,情報処理学会論文誌,Vol.540,No.2,pp.630-638(2013).
 
% [2]吉井英人,乃村能成,谷口秀夫:作業発生の規則性に基づく作業予測手法,マルチメディア通信と分散処理ワークショップ論文集, vol.2012, no.4, pp.58-64 (2012).


\end{document}
