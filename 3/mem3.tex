\documentclass[fleqn, 14pt]{extarticlej}
\oddsidemargin=-1cm
\textwidth=18cm
\textheight=23cm
\topmargin=0cm
\headheight=1cm
\headsep=0cm
\footskip=0.8cm

\pagestyle{plain}
\usepackage{ulem}
\usepackage{indentfirst}

\renewcommand\labelenumi{(\theenumi)}
\def\descriptionlabel#1{\hspace\labelsep #1}

\makeatletter
\def\section{\@startsection {section}{1}{\z@}%
{-3.5ex \@plus -1ex \@minus -.2ex}%
{2.3ex \@plus.2ex}%
{\sectionformat}}
\def\sectionformat{\normalsize}
\makeatother

\def\l@section{\@dottedtocline{1}{2.5em}{2.3em}}
% 最初の 1 は toc の深さの定義、1.5em は左端からの字下げ幅、
% 2.3em は番号と見出しの間隔
\def\l@subsection{\@dottedtocline{2}{3.8em}{3.2em}}
 
\begin{document}
\setcounter{section}{-1}

%%%%%%%%%%%%%%%%%%%%%%%%%%%%
%% 表題
%%%%%%%%%%%%%%%%%%%%%%%%%%%%
\begin{center}
  {\Large {\bf 記録書} No.3}

  (2013年 5月 8日〜 5月 17日)
\end{center}
\begin{flushright}
  2013年 5月 20日\\
  乃村研究室\\
  河野 達生
\end{flushright}

\vspace{-1.5em}
%%%%%%%%%%%%%%%%%%%%%%%%%%%%
%% 前回ミーティングでの指導・指摘事項
%%%%%%%%%%%%%%%%%%%%%%%%%%%%
\section*{0. 前回ミーティングでの指導・指摘事項}
\vspace{-1.0em}
\begin{enumerate}
\item 「研究の幅を広げる」という表現において,実際にやってみるという意味では適切ではない.
	特徴や特性を理解し検討することが重要である.
	\begin{flushright}
		[5/8,第126回 GN検討打合せ,乃村先生]
	\end{flushright}
\end{enumerate}


\vspace{-1.5em}
%%%%%%%%%%%%%%%%%%%%%%%%%%%%
%% 実績
%%%%%%%%%%%%%%%%%%%%%%%%%%%%
\section*{1. 実績}
\vspace{-1.0em}

\begin{enumerate}
  \itemsep -1mm

  \item[ 1.1] 研究関連

    \begin{tabular}{p{11.8cm}l}
      (1) EnCalの概要把握&(10%,+10%)

    \end{tabular}

  \item[ 1.2] 研究室関連

    \begin{tabular}{p{11.8cm}l}
	  (1) 第126回GN検討打合せ&(5/8)\\
	  (2) 乃村研ミーティング&(5/8)\\
	  (3) 第5回ノムニチ開発進捗報告会&(5/8)\\
      (4) 第70回GN開発打合せ&(5/10)\\
      (5) 第3回TOEIC勉強会&(5/14)\\
	  
    \end{tabular}

  \item[ 1.3] 大学関連

    \begin{tabular}{p{11.8cm}l}
      (1) 情報化における職業&(5/10)\\
    \end{tabular}
    
  \item[ 1.4] 就職関連

    \begin{tabular}{p{11.8cm}l}
	  (1)三菱UFJインフォメーションテクノロジー株式会社\\
	 (一次選考)&(5/11)\\
	  (2)株式会社JSOL(一次選考)&(5/16)\\
    \end{tabular}
    
\end{enumerate}

\vspace{-1.5em}
%%%%%%%%%%%%%%%%%%%%%%%%%%%%
%% 詳細と感想
%%%%%%%%%%%%%%%%%%%%%%%%%%%%
\section*{2. 詳細と感想}
\vspace{-1.0em}

\begin{description}
  \itemsep -1mm

\item[ 2.1] 研究関連
  \vspace{-0.8em}
  \begin{enumerate}
    \renewcommand{\labelenumi}{(1)}
  \item 研究テーマが情報の可視化,整理に決まった.例えば,EnCalを使いミッションを軸として並べなおすとどうなるかを研究する.EnCalとは,作業の周期性や関連性を扱うカレンダシステムである.このため,EnCalを使用し概要を把握する.吉井さんからEnCalの環境構築の資料を頂き,資料に従い実行する.
  
  
  \end{enumerate}

  \vspace{-0.5em}
  \item[ 2.2] 研究室関連
    \vspace{-1.0em}
    \begin{enumerate}
      \renewcommand{\labelenumi}{(2)}
    \item 
乃村研ミーティングに参加した.乃村研では,毎年乃村研メンバ全員で文献を読み理解を深める勉強会が行われる.今年の文献は,「Bad Data Handbook」[1]である.英語の文献であるため,英語の勉強も一緒に行う.
      \renewcommand{\labelenumi}{(5)}
    \item 第3回TOEIC勉強会に参加した.模擬テストを行うことで,TOEICのテスト形式に慣れてきた.今後は語彙を増やし,文章読解の得点を上げる.
    
    \end{enumerate}

	%\vspace{-0.5em}
%% \item[ 2.3] 大学関連
%% 	\vspace{-0.8em}
%% 	\begin{enumerate}
%% 	  \renewcommand{\labelenumi}{(1)}
%% 	\item 計画停電の復旧作業に参加した.
%% \end{enumerate}
\end{description}

\vspace{-1.5em}
%%%%%%%%%%%%%%%%%%%%%%%%%%%
%% 予定
%%%%%%%%%%%%%%%%%%%%%%%%%%%%
\section*{3. 予定}
\vspace{-0.8em}

\begin{description}
  \itemsep -1mm
\item[ 3.1] 研究関連

  \begin{tabular}{p{11.8cm}l}
    (1) EnCalの概要把握&(5/30)\\
     
  \end{tabular}

\item[ 3.2] 研究室関連

  \begin{tabular}{p{11.8cm}l}
    (1) 第71回GN開発打合せ&(5/29)\\
	(2) 第127回GN検討打合せ&(5/30)\\
	(3) 平成25年度乃村研勉強会&(6/3)\\
	(4) 乃村研ミーティング&(6/3)\\
	(5) 第6回ノムニチ進捗開発報告会&(6/3)\\
  \end{tabular}

\item[ 3.3] 大学関連

  \begin{tabular}{p{11.8cm}l}
    (1) 第180回TOEIC公開テスト&(5/26)\\
    (2) カレッジTOEIC&(6/1)\\
  \end{tabular}

\end{description}

%\vspace{-1.5em}
%%%%%%%%%%%%%%%%%%%%%%%%%%%%
%% その他
%%%%%%%%%%%%%%%%%%%%%%%%%%%%
\section*{4. その他}
\vspace{-1.0em}
5月12日(日)に自身が所属するアルティメットフリスビーサークルで大会に出場した.大会の開催地は,倉敷市にある福田公園であった.3月に卒業された先輩方も参加しており,先輩方のプレーが懐かしく感じた.私の所属したチームは,準優勝した.時間を作り,夏の全国大会出場に向けて練習を再開する.
\vspace{-1.0em}
\section*{5. 関連文献}
\vspace{-1.0em}
	[1] McCallum, E.Q.: Bad Data Handbook,O'REILLY(2012).
\end{document}
